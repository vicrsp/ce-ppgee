
%% bare_conf.tex
%% V1.4b
%% 2015/08/26
%% by Michael Shell
%% See:
%% http://www.michaelshell.org/
%% for current contact information.
%%
%% This is a skeleton file demonstrating the use of IEEEtran.cls
%% (requires IEEEtran.cls version 1.8b or later) with an IEEE
%% conference paper.
%%
%% Support sites:
%% http://www.michaelshell.org/tex/ieeetran/
%% http://www.ctan.org/pkg/ieeetran
%% and
%% http://www.ieee.org/

%%*************************************************************************
%% Legal Notice:
%% This code is offered as-is without any warranty either expressed or
%% implied; without even the implied warranty of MERCHANTABILITY or
%% FITNESS FOR A PARTICULAR PURPOSE! 
%% User assumes all risk.
%% In no event shall the IEEE or any contributor to this code be liable for
%% any damages or losses, including, but not limited to, incidental,
%% consequential, or any other damages, resulting from the use or misuse
%% of any information contained here.
%%
%% All comments are the opinions of their respective authors and are not
%% necessarily endorsed by the IEEE.
%%
%% This work is distributed under the LaTeX Project Public License (LPPL)
%% ( http://www.latex-project.org/ ) version 1.3, and may be freely used,
%% distributed and modified. A copy of the LPPL, version 1.3, is included
%% in the base LaTeX documentation of all distributions of LaTeX released
%% 2003/12/01 or later.
%% Retain all contribution notices and credits.
%% ** Modified files should be clearly indicated as such, including  **
%% ** renaming them and changing author support contact information. **
%%*************************************************************************


% *** Authors should verify (and, if needed, correct) their LaTeX system  ***
% *** with the testflow diagnostic prior to trusting their LaTeX platform ***
% *** with production work. The IEEE's font choices and paper sizes can   ***
% *** trigger bugs that do not appear when using other class files.       ***                          ***
% The testflow support page is at:
% http://www.michaelshell.org/tex/testflow/



\documentclass[conference]{IEEEtran}
\usepackage[brazil]{babel}
% Some Computer Society conferences also require the compsoc mode option,
% but others use the standard conference format.
%
% If IEEEtran.cls has not been installed into the LaTeX system files,
% manually specify the path to it like:
% \documentclass[conference]{../sty/IEEEtran}





% Some very useful LaTeX packages include:
% (uncomment the ones you want to load)


% *** MISC UTILITY PACKAGES ***
%
%\usepackage{ifpdf}
% Heiko Oberdiek's ifpdf.sty is very useful if you need conditional
% compilation based on whether the output is pdf or dvi.
% usage:
% \ifpdf
%   % pdf code
% \else
%   % dvi code
% \fi
% The latest version of ifpdf.sty can be obtained from:
% http://www.ctan.org/pkg/ifpdf
% Also, note that IEEEtran.cls V1.7 and later provides a builtin
% \ifCLASSINFOpdf conditional that works the same way.
% When switching from latex to pdflatex and vice-versa, the compiler may
% have to be run twice to clear warning/error messages.






% *** CITATION PACKAGES ***
%
%\usepackage{cite}
% cite.sty was written by Donald Arseneau
% V1.6 and later of IEEEtran pre-defines the format of the cite.sty package
% \cite{} output to follow that of the IEEE. Loading the cite package will
% result in citation numbers being automatically sorted and properly
% "compressed/ranged". e.g., [1], [9], [2], [7], [5], [6] without using
% cite.sty will become [1], [2], [5]--[7], [9] using cite.sty. cite.sty's
% \cite will automatically add leading space, if needed. Use cite.sty's
% noadjust option (cite.sty V3.8 and later) if you want to turn this off
% such as if a citation ever needs to be enclosed in parenthesis.
% cite.sty is already installed on most LaTeX systems. Be sure and use
% version 5.0 (2009-03-20) and later if using hyperref.sty.
% The latest version can be obtained at:
% http://www.ctan.org/pkg/cite
% The documentation is contained in the cite.sty file itself.






% *** GRAPHICS RELATED PACKAGES ***
%
\ifCLASSINFOpdf
   \usepackage[pdftex]{graphicx}
  % declare the path(s) where your graphic files are
  % \graphicspath{{../pdf/}{../jpeg/}}
  % and their extensions so you won't have to specify these with
  % every instance of \includegraphics
  % \DeclareGraphicsExtensions{.pdf,.jpeg,.png}
\else
  % or other class option (dvipsone, dvipdf, if not using dvips). graphicx
  % will default to the driver specified in the system graphics.cfg if no
  % driver is specified.
  % \usepackage[dvips]{graphicx}
  % declare the path(s) where your graphic files are
  % \graphicspath{{../eps/}}
  % and their extensions so you won't have to specify these with
  % every instance of \includegraphics
  % \DeclareGraphicsExtensions{.eps}
\fi
% graphicx was written by David Carlisle and Sebastian Rahtz. It is
% required if you want graphics, photos, etc. graphicx.sty is already
% installed on most LaTeX systems. The latest version and documentation
% can be obtained at: 
% http://www.ctan.org/pkg/graphicx
% Another good source of documentation is "Using Imported Graphics in
% LaTeX2e" by Keith Reckdahl which can be found at:
% http://www.ctan.org/pkg/epslatex
%
% latex, and pdflatex in dvi mode, support graphics in encapsulated
% postscript (.eps) format. pdflatex in pdf mode supports graphics
% in .pdf, .jpeg, .png and .mps (metapost) formats. Users should ensure
% that all non-photo figures use a vector format (.eps, .pdf, .mps) and
% not a bitmapped formats (.jpeg, .png). The IEEE frowns on bitmapped formats
% which can result in "jaggedy"/blurry rendering of lines and letters as
% well as large increases in file sizes.
%
% You can find documentation about the pdfTeX application at:
% http://www.tug.org/applications/pdftex





% *** MATH PACKAGES ***
%
%\usepackage{amsmath}
% A popular package from the American Mathematical Society that provides
% many useful and powerful commands for dealing with mathematics.
%
% Note that the amsmath package sets \interdisplaylinepenalty to 10000
% thus preventing page breaks from occurring within multiline equations. Use:
%\interdisplaylinepenalty=2500
% after loading amsmath to restore such page breaks as IEEEtran.cls normally
% does. amsmath.sty is already installed on most LaTeX systems. The latest
% version and documentation can be obtained at:
% http://www.ctan.org/pkg/amsmath





% *** SPECIALIZED LIST PACKAGES ***
%
%\usepackage{algorithmic}
% algorithmic.sty was written by Peter Williams and Rogerio Brito.
% This package provides an algorithmic environment fo describing algorithms.
% You can use the algorithmic environment in-text or within a figure
% environment to provide for a floating algorithm. Do NOT use the algorithm
% floating environment provided by algorithm.sty (by the same authors) or
% algorithm2e.sty (by Christophe Fiorio) as the IEEE does not use dedicated
% algorithm float types and packages that provide these will not provide
% correct IEEE style captions. The latest version and documentation of
% algorithmic.sty can be obtained at:
% http://www.ctan.org/pkg/algorithms
% Also of interest may be the (relatively newer and more customizable)
% algorithmicx.sty package by Szasz Janos:
% http://www.ctan.org/pkg/algorithmicx




% *** ALIGNMENT PACKAGES ***
%
%\usepackage{array}
% Frank Mittelbach's and David Carlisle's array.sty patches and improves
% the standard LaTeX2e array and tabular environments to provide better
% appearance and additional user controls. As the default LaTeX2e table
% generation code is lacking to the point of almost being broken with
% respect to the quality of the end results, all users are strongly
% advised to use an enhanced (at the very least that provided by array.sty)
% set of table tools. array.sty is already installed on most systems. The
% latest version and documentation can be obtained at:
% http://www.ctan.org/pkg/array


% IEEEtran contains the IEEEeqnarray family of commands that can be used to
% generate multiline equations as well as matrices, tables, etc., of high
% quality.




% *** SUBFIGURE PACKAGES ***
%\ifCLASSOPTIONcompsoc
%  \usepackage[caption=false,font=normalsize,labelfont=sf,textfont=sf]{subfig}
%\else
%  \usepackage[caption=false,font=footnotesize]{subfig}
%\fi
% subfig.sty, written by Steven Douglas Cochran, is the modern replacement
% for subfigure.sty, the latter of which is no longer maintained and is
% incompatible with some LaTeX packages including fixltx2e. However,
% subfig.sty requires and automatically loads Axel Sommerfeldt's caption.sty
% which will override IEEEtran.cls' handling of captions and this will result
% in non-IEEE style figure/table captions. To prevent this problem, be sure
% and invoke subfig.sty's "caption=false" package option (available since
% subfig.sty version 1.3, 2005/06/28) as this is will preserve IEEEtran.cls
% handling of captions.
% Note that the Computer Society format requires a larger sans serif font
% than the serif footnote size font used in traditional IEEE formatting
% and thus the need to invoke different subfig.sty package options depending
% on whether compsoc mode has been enabled.
%
% The latest version and documentation of subfig.sty can be obtained at:
% http://www.ctan.org/pkg/subfig




% *** FLOAT PACKAGES ***
%
%\usepackage{fixltx2e}
% fixltx2e, the successor to the earlier fix2col.sty, was written by
% Frank Mittelbach and David Carlisle. This package corrects a few problems
% in the LaTeX2e kernel, the most notable of which is that in current
% LaTeX2e releases, the ordering of single and double column floats is not
% guaranteed to be preserved. Thus, an unpatched LaTeX2e can allow a
% single column figure to be placed prior to an earlier double column
% figure.
% Be aware that LaTeX2e kernels dated 2015 and later have fixltx2e.sty's
% corrections already built into the system in which case a warning will
% be issued if an attempt is made to load fixltx2e.sty as it is no longer
% needed.
% The latest version and documentation can be found at:
% http://www.ctan.org/pkg/fixltx2e


%\usepackage{stfloats}
% stfloats.sty was written by Sigitas Tolusis. This package gives LaTeX2e
% the ability to do double column floats at the bottom of the page as well
% as the top. (e.g., "\begin{figure*}[!b]" is not normally possible in
% LaTeX2e). It also provides a command:
%\fnbelowfloat
% to enable the placement of footnotes below bottom floats (the standard
% LaTeX2e kernel puts them above bottom floats). This is an invasive package
% which rewrites many portions of the LaTeX2e float routines. It may not work
% with other packages that modify the LaTeX2e float routines. The latest
% version and documentation can be obtained at:
% http://www.ctan.org/pkg/stfloats
% Do not use the stfloats baselinefloat ability as the IEEE does not allow
% \baselineskip to stretch. Authors submitting work to the IEEE should note
% that the IEEE rarely uses double column equations and that authors should try
% to avoid such use. Do not be tempted to use the cuted.sty or midfloat.sty
% packages (also by Sigitas Tolusis) as the IEEE does not format its papers in
% such ways.
% Do not attempt to use stfloats with fixltx2e as they are incompatible.
% Instead, use Morten Hogholm'a dblfloatfix which combines the features
% of both fixltx2e and stfloats:
%
% \usepackage{dblfloatfix}
% The latest version can be found at:
% http://www.ctan.org/pkg/dblfloatfix




% *** PDF, URL AND HYPERLINK PACKAGES ***
%
%\usepackage{url}
% url.sty was written by Donald Arseneau. It provides better support for
% handling and breaking URLs. url.sty is already installed on most LaTeX
% systems. The latest version and documentation can be obtained at:
% http://www.ctan.org/pkg/url
% Basically, \url{my_url_here}.




% *** Do not adjust lengths that control margins, column widths, etc. ***
% *** Do not use packages that alter fonts (such as pslatex).         ***
% There should be no need to do such things with IEEEtran.cls V1.6 and later.
% (Unless specifically asked to do so by the journal or conference you plan
% to submit to, of course. )


% correct bad hyphenation here
\hyphenation{op-tical net-works semi-conduc-tor}


\begin{document}
%
% paper title
% Titles are generally capitalized except for words such as a, an, and, as,
% at, but, by, for, in, nor, of, on, or, the, to and up, which are usually
% not capitalized unless they are the first or last word of the title.
% Linebreaks \\ can be used within to get better formatting as desired.
% Do not put math or special symbols in the title.
\title{Computação Evolucionária \\ Trabalho Prático 1}


% author names and affiliations
% use a multiple column layout for up to three different
% affiliations
\author{\IEEEauthorblockN{Victor São Paulo Ruela}
\IEEEauthorblockA{Programa de Pós-Graduação em Engenharia Elétrica\\
Universidade Federal de Minas Gerais\\
Belo Horizonte, Brasil\\
Email: victorspruela@gmail.com}}


% conference papers do not typically use \thanks and this command
% is locked out in conference mode. If really needed, such as for
% the acknowledgment of grants, issue a \IEEEoverridecommandlockouts
% after \documentclass

% for over three affiliations, or if they all won't fit within the width
% of the page, use this alternative format:
% 
%\author{\IEEEauthorblockN{Michael Shell\IEEEauthorrefmark{1},
%Homer Simpson\IEEEauthorrefmark{2},
%James Kirk\IEEEauthorrefmark{3}, 
%Montgomery Scott\IEEEauthorrefmark{3} and
%Eldon Tyrell\IEEEauthorrefmark{4}}
%\IEEEauthorblockA{\IEEEauthorrefmark{1}School of Electrical and Computer Engineering\\
%Georgia Institute of Technology,
%Atlanta, Georgia 30332--0250\\ Email: see http://www.michaelshell.org/contact.html}
%\IEEEauthorblockA{\IEEEauthorrefmark{2}Twentieth Century Fox, Springfield, USA\\
%Email: homer@thesimpsons.com}
%\IEEEauthorblockA{\IEEEauthorrefmark{3}Starfleet Academy, San Francisco, California 96678-2391\\
%Telephone: (800) 555--1212, Fax: (888) 555--1212}
%\IEEEauthorblockA{\IEEEauthorrefmark{4}Tyrell Inc., 123 Replicant Street, Los Angeles, California 90210--4321}}




% use for special paper notices
%\IEEEspecialpapernotice{(Invited Paper)}




% make the title area
\maketitle

% As a general rule, do not put math, special symbols or citations
% in the abstract
\begin{abstract}
O objetivo deste trabalho é estudar a aplicação de algoritmos genéticos para minimização da função de Rastrigin e a solução do problema das N-Rainhas. Dessa forma, é possível ver como essa classe de algoritmos se comporta para dois tipos diferentes de problema de otimização, bem como os desafios de sua implementação e ajuste dos seus hiper-parâmetros.

\end{abstract}

% no keywords




% For peer review papers, you can put extra information on the cover
% page as needed:
% \ifCLASSOPTIONpeerreview
% \begin{center} \bfseries EDICS Category: 3-BBND \end{center}
% \fi
%
% For peerreview papers, this IEEEtran command inserts a page break and
% creates the second title. It will be ignored for other modes.
%\IEEEpeerreviewmaketitle



\section{Introdução}

O problema das N-Rainhas \cite{bell2009survey} consiste em posicionar N rainhas num tabuleiro de xadrez regular (NxN), de forma que nenhuma rainha seja capaz de capturar outra rainha. Portanto, uma solução requer que não existam duas rainhas que compartilhem a mesma linha, coluna ou diagonal no tabuleiro. Para 8 rainhas, o problema já se torna computacionalmente custoso, uma vez que existem mais de 4 bilhões de organizações possíves para somente 92 soluções. 

A função de Rastrigin \cite{hoffmeister1991} é uma função não-convexa cujo mínimo global é difícil de ser encontrado devido ao seu grande espaço de busca e quantidade de mínimos locais, sendo utilizada para avaliar o desempenho de algoritmos de otimização. Para duas variáveis, as curvas de nível são exibidas na Figura \ref{fig:rastrigin2d}.

\begin{figure}[h!]
	\centering
	\includegraphics[width=2.5in]{rastrigin2d.png}
	\caption{Curvas de níveis para função de Rastrigin em duas dimensões}
	\label{fig:rastrigin2d}
\end{figure}

\section{Metodologia}

A primeira etapa consiste na implementação da funções objetivo de cada problema e dos algoritmos genéticos que serão utilizados para sua solução, o que será feito na linguagem Python (versão 3.8.5). A validação da implementação será feita sobre funções de teste simples presentes na literatura. 

Será considerada inicialmente uma instância padrão dos algoritmos para cada problema e avaliados os principais hiper-parâmetros para entender o impacto de cada um sobre o desempenho do algoritmo. Baseado nos resultados obtidos, novos operadores ou estratégias para os algoritmos serão experimentadas com o objetivo de melhorar o desempenho da instância padrão, descritas a seguir.

\subsection{Problema das N-Rainhas}

\subsubsection{{Representação}}
O genótipo será representado por permutação de inteiros, utilizando um vetor com o tamanho do número de colunas, onde cada valor será um inteiro de 1 a N, indicando em qual linha a rainha está posicionada. Além disso, não será permitido ter números repetidos, de forma a eliminar as soluções inválidas de rainhas na mesma linha, reduzindo o espaço de busca do algoritmo. 

\subsubsection{{Critérios de parada}}
Será considerado como critério a convergência para uma solução ótima, ou um número máximo de 300 gerações.

\subsubsection{População inicial}
A população inicial será obtida aleatoriamente de acordo com o tamanho da população escolhida. Seu valor padrão será de 100 indivíduos.

\subsubsection{Função de escalonamento}
Dado que na representação adotada o número máximo de cheques em um tabuleiro é dado por $q_{max} = N(N-1)/2$, para $N=8$ este número será 28. Como a solução do problema consiste em minimizar o número de cheques, logo a função de fitness utilizada seguirá a método de \textit{shift}:

\begin{equation}
fitness=q_{max} - q(f)
\end{equation}

\subsubsection{Operador de Cruzamento}
Foi utilizado o operador \textit{Order Crossover}, o qual realiza os seguintes passos \cite{eibenbook}:
\begin{enumerate}
	\item Escolher dois pontos de cruzamento aleatoriamente e copiar o segmento entre eles do primeiro pai no primeiro filho
	\item Iniciando do segundo ponto de cruzamento no segundo pai, copiar o restante dos números não utilizadas no primeiro filho na ordem que eles aparecem
	\item Criar o segundo filho analogamente, porém invertendo a ordem dos pais.
\end{enumerate}  
Dois indivíduos são escolhidos aleatoriamente da população para o cruzamento com uma determinada probabilidade, cujo valor padrão é de 0.6.

\subsubsection{Operador de Mutação}
Será utilizado o operador \textit{Swap Mutation}. Nele, duas variáveis de um indivíduo são escolhidas aleatoriamente e trocadas de posição. Em cada geração, a mutação poderá ocorrer para cada indivíduo com uma determinada probabilidade, cujo valor padrão é de 0.1.

\subsubsection{Seleção dos pais}
Dois indivíduos são escolhidos utilizando a técnica de amostragem estocástica universal. Ela é uma extensão do algoritmo da roleta, com a diferença de que ao invés de girar a roleta $\lambda$ vezes, esta é divida em $\lambda$ espaços iguais e girada somente uma vez. É mais indicada quando queremos amostrar mais de um indivíduo de uma mesma população \cite{eibenbook}.

\subsubsection{Seleção dos sobreviventes}
Os dois filhos gerados são incluídos na população e são eliminados os dois indivíduos com o pior fitness. 

\subsection{Função de Rastrigin}

\subsubsection{{Representação}}
O genótipo será representado usando código de gray com $L = 20$ bits por variável de decisão. Como o problema possui muitos mínimos locais, o fenômeno de \textit{Hamming Cliffs} pode afetar o desempenho do algoritmo, o que motivou o uso da codificação de gray em relação à binária.

\subsubsection{{Critérios de parada}}
Será considerado como critério de parada 10000 avaliações da função objetivo.

\subsubsection{{População inicial}}
A população inicial será obtida aleatoriamente de acordo com o tamanho da população escolhido. O tamanho padrão será de 100 indivíduos.

\subsubsection{{Função de escalonamento}}
Dado que este é um problema de minimização e o mínimo da função de rastrigin possui valor 0, a técnica de inversão será utilizada para definir a função de fitness da seguinte forma:

\begin{equation}
fitness= \frac{1}{f(x) + 1}
\end{equation}

\subsubsection{{Operador de Cruzamento}}
Será utilizado o operador de 1 ponto de corte por variável. Pares de indivíduos são escolhidos aleatoriamente na população, sem repetição. O cruzamento é realizado para cada par com uma determinada probabilidade, cujos valores serão adaptados linearmente a cada geração. A faixa padrão na qual os valores variam é de 0.6 a 0.9.

\subsubsection{{Operador de Mutação}}
Será utilizado o operador bit fip. A mutação poderá ocorrer por bit e para cada variável, com uma probabilidade variando linearmente de acordo com a geração. A faixa padrão na qual os valores variam é de 0.05 a 0.01.

\subsubsection{{Seleção dos pais}}
Em cada geração, o operador de seleção poderá ser o da roleta ou torneio. É sorteado um número no intervalo de 0 a 1, e se o valor é menor que 0.5, a seleção será por roleta, caso contrário será por torneio. Serão considerados 10 candidatos aleatórios da população para a seleção por torneio. 

\subsubsection{{Seleção dos sobreviventes}}
Será utilizada uma abordagem geracional, logo esta etapa não será necessária.


\section{Resultados}

\subsection{Problema das N-Rainhas}

\subsubsection{Efeito do tamanho da população}

Para avaliar o efeito do tamanho da população, executou-se o algoritmo para populações de 10 e 100 indivíduos. A distribuição dos resultados para 30 execuções de cada instância podem ser vistos na Figura \ref{fig:queens_scenario_pop_variation}. É possível notar que para a população menor, a distribuição da geração de convergêcia concentra-se no valor máximo de gerações, ou seja, o algoritmo não convergiu na maioria das execuções. De fato, 18 execuções desta instância convergiram, contra 22 da com 100 indivíduos. Logo, pode-se concluir que populações maiores favorecem a convergência do algoritmo. 

Um resultado inesperado que ocorreu para a instância com população de 100 indivíviduos está no fato de que em 13 vezes uma solução ótima era encontrada na população inicial. Isso pode ter sido o resultado de uma boa escolha do genótipo para representar o problema, ou simplesmente sorte. 

\begin{figure}[h!]
	\centering
	\includegraphics[width=2.5in]{queens_scenario_pop_variation_crop.png}
	\caption{Histograma da geração de convergência para teste de variação do tamanho da população}
	\label{fig:queens_scenario_pop_variation}
\end{figure}

\subsubsection{Efeito da probabilidade de crossover}

Para estudar o efeito de crossover na solução do algoritmo, foram avaliados os valores de probabilidade 0.3 e 0.9. A mutação foi desativada para as duas instâncias. Os resultados para 30 execuções de cada instância podem ser vistos na Figura \ref{fig:queens_scenario_crossover_variation}. É possível ver também que o crossover é importante para a convergência do algoritmo, uma vez que todas as execuções para a instância de probabilidade de 0.9 convergem, entretanto para a instância com probabilidade de 0.3, 14 não conseguem chegar a uma solução ótima com menos de 300 gerações.

\begin{figure}[h!]
	\centering
	\includegraphics[width=2.5in]{queens_scenario_crossover_variation_v2.png}
	\caption{Histograma da geração de convergência para teste de variação da probabilidade de crossover}
	\label{fig:queens_scenario_crossover_variation}
\end{figure}

\subsubsection{Efeito da probabilidade de mutação}

Para estudar o efeito da mutação na solução do algoritmo, foram avaliados os valores de probabilidade 0.001 e 0.2. Os resultados para 30 execuções de cada instância podem ser vistos na Figura \ref{fig:queens_scenario_mutation_variation}. Note que a probabilidade de mutação é muito importante para a convergência do algoritmo, uma vez que para a instância com probabilidade de 0.001 somente 6 instâncias convergem, equanto para a instância com probabildiade 0.2 isso ocorre para 16 execuções. 

\begin{figure}[h!]
	\centering
	\includegraphics[width=2.5in]{queens_scenario_mutation_variation_v2.png}
	\caption{Histograma da geração de convergência para teste da variação da probabilidade de mutação}
	\label{fig:queens_scenario_mutation_variation}
\end{figure}

%\begin{figure}[h!]
%	\centering
%	\includegraphics[width=2.5in]{queens_scenario_mutation_variation_average.png}
%	\caption{Perfil de convergência da solução média para variação da probabilidade de mutação}
%	\label{fig:queens_scenario_mutation_variation_average}
%\end{figure}

\subsubsection{Mutação por \textit{Inversion Mutation}}

Conforme visto anteriormente, diferentes probabilidades de mutação tem grande impacto no desempenho do algoritmo. Portanto, foi proposto como melhoria o uso do operador \textit{Inversion Mutation}, o qual funciona da seguinte forma:
\begin{enumerate}
	\item Ecolher dois índices aleatoriamente
	\item Inverter a ordem dos elementos entre eles
\end{enumerate}

Os resultados da comparação com o operador de \textit{Swap Mutation} são exibidos na Figura \ref{fig:queens_scenario_mutation_inversion}. Para 30 execuções do algoritmo, ele convergiu 24 vezes para \textit{Swap Mutation} e 25 para \textit{Inversion Mutation}. Desconsiderando as convergências com menos de 25 gerações, às quais estão bastante relacionadas com a obtenção de uma população inicial que já contém a solução ótima ou uma bem próxima, que é um processo aleatório, o novo operador proposto aparenta possuir melhor desempenho, uma vez que possui maior quantidade de execuções que convergiram abaixo de 175 gerações. 

\begin{figure}[h!]
	\centering
	\includegraphics[width=2.5in]{queens_scenario_mutation_inversion_crop.png}
	\caption{Histograma da geração de convergência para operador de \textit{Inversion Mutation} em relação ao \textit{Swap Mutation}}
	\label{fig:queens_scenario_mutation_inversion}
\end{figure}


\subsection{Função de Rastrigin}

Os gráficos referentes aos testes sobre essa função apresentam somente os dados das 50 gerações iniciais, uma vez que o algoritmo convergia muito rápido para um valor próximo de zero, dificultando a comparação dos resultados para todas as gerações disponíveis.

\subsubsection{Efeito do número de bits da representação}
Para avaliar o efeito de bits utilizados na representação, executou-se o algoritmo para os valores de 10 e 30 bits, mantendo os demais hiper-parâmetros fixos nos valores padrão. Os resultados para 30 execuções de cada instância podem ser vistos na Figura \ref{fig:scenario_bits_variation}. 

\begin{figure}[h!]
	\centering
	\includegraphics[width=2.5in]{scenario_bits_variation_v2.png}
	\caption{Perfil de convergência das execuções para a variação do número de bits}
	\label{fig:scenario_bits_variation}
\end{figure}

Para 10 bits, o função objetivo em média converge para 0.05, enquanto para 30 bits este valor é de aproximadamente 1.42x$10^{-14}$. Logo, é possível concluir que o número de bits afeta bastante a melhor solução que o algoritmo consegue obter.

\subsubsection{Efeito do tamanho da população}

Para avaliar o efeito do tamanho da população, executou-se o algoritmo para populações de 10 e 100 indivíduos, utilizando os demais parâmetro da instância padrão. Os resultados para 30 execuções de cada instância podem ser vistos na Figura \ref{fig:scenario_pop_variation}. É interessante notar que para os dois tamanhos de população, a convergência para um valor próximo de zero é bem rápida, entretanto para uma população maior a função objetivo apresenta um decréscimo mais estável.

\begin{figure}[h!]
	\centering
	\includegraphics[width=2.5in]{scenario_pop_variation_v2.png}
	\caption{Perfil de convergência das execuções para a variação do tamanho da população}
	\label{fig:scenario_pop_variation}
\end{figure}

\subsubsection{Efeito da probabilidade de crossover}

Para estudar o efeito de crossover na solução do algoritmo, foram avaliadas duas instâncias com intervalos $[0.6,0.8]$ e $[0.8, 1.0]$, dado que está sendo utilizada a abordagem por adaptação linear dos parâmetros a cada geração. A probabilidade de mutação foi desativada para os testes. Os resultados para 30 execuções de cada instância podem ser vistos na Figura \ref{fig:scenario_crossover_variation}. É possível notar que os resultados são bem similares para as duas instâncias avaliadas.

\begin{figure}[h!]
	\centering
	\includegraphics[width=2.5in]{scenario_crossover_variation_v2.png}
	\caption{Perfil de convergência das execuções para variação das faixas de probabilidade de crossover}
	\label{fig:scenario_crossover_variation}
\end{figure}

\subsubsection{Efeito da probabilidade de mutação}

Para estudar o efeito de crossover na solução do algoritmo, foram avaliadas duas instâncias com os intervalos $[0.6,0.8]$ e $[0.8, 1.0]$, dado que está sendo utilizada a abordagem por adaptação linear dos parâmetros a cada geração. A probabilidade de crossover foi desativada para os testes. Os resultados para 30 execuções de cada instância podem ser vistos na Figura \ref{fig:scenario_mutation_variation}. É possível notar que os resultados são bem similares para as duas instâncias avaliadas, não havendo como diferenciar o desempenho entre elas.

\begin{figure}[h!]
	\centering
	\includegraphics[width=2.5in]{scenario_mutation_variation_v2.png}
	\caption{Perfil de convergência das execuções para variação das faixas de probabilidade de mutação}
	\label{fig:scenario_mutation_variation}
\end{figure}

\subsubsection{Aplicação de escala linear ao fitness}
O objetivo desta possível melhoria é fazer com que a função de fitness represente melhor a função objetivo próximo de seu mínimo global, uma vez que a função de escala utilizada não consegue diferenciar muito bem valores próximos de zero. Utilizando o fator de escala $K = 1.5$, os resultados para 30 execuções das instâncias com e sem escala linear podem ser vistos na Figura \ref{fig:scenario_linear_scaling}.

\begin{figure}[h!]
	\centering
	\includegraphics[width=2.5in]{scenario_linear_scaling_v2.png}
	\caption{Resultados do uso de escala linear}
	\label{fig:scenario_linear_scaling}
\end{figure}

É interessante notar que em média, o algoritmo possui uma convergência mais rápida e estável quando não é usada escala linear. Um resultado indesejado durante a execução, porém esperado, foi que alguns valores abaixo de zero eram obtidos após a aplicação da escala linear em algumas gerações. Isso não afeta o funcionamento do algoritmo, porém poderiam causar resultados indesejados na geração dos gráficos. 



% An example of a floating figure using the graphicx package.
% Note that \label must occur AFTER (or within) \caption.
% For figures, \caption should occur after the \includegraphics.
% Note that IEEEtran v1.7 and later has special internal code that
% is designed to preserve the operation of \label within \caption
% even when the captionsoff option is in effect. However, because
% of issues like this, it may be the safest practice to put all your
% \label just after \caption rather than within \caption{}.
%
% Reminder: the "draftcls" or "draftclsnofoot", not "draft", class
% option should be used if it is desired that the figures are to be
% displayed while in draft mode.
%
%\begin{figure}[!t]
%\centering
%\includegraphics[width=2.5in]{myfigure}
% where an .eps filename suffix will be assumed under latex, 
% and a .pdf suffix will be assumed for pdflatex; or what has been declared
% via \DeclareGraphicsExtensions.
%\caption{Simulation results for the network.}
%\label{fig_sim}
%\end{figure}

% Note that the IEEE typically puts floats only at the top, even when this
% results in a large percentage of a column being occupied by floats.


% An example of a double column floating figure using two subfigures.
% (The subfig.sty package must be loaded for this to work.)
% The subfigure \label commands are set within each subfloat command,
% and the \label for the overall figure must come after \caption.
% \hfil is used as a separator to get equal spacing.
% Watch out that the combined width of all the subfigures on a 
% line do not exceed the text width or a line break will occur.
%
%\begin{figure*}[!t]
%\centering
%\subfloat[Case I]{\includegraphics[width=2.5in]{box}%
%\label{fig_first_case}}
%\hfil
%\subfloat[Case II]{\includegraphics[width=2.5in]{box}%
%\label{fig_second_case}}
%\caption{Simulation results for the network.}
%\label{fig_sim}
%\end{figure*}
%
% Note that often IEEE papers with subfigures do not employ subfigure
% captions (using the optional argument to \subfloat[]), but instead will
% reference/describe all of them (a), (b), etc., within the main caption.
% Be aware that for subfig.sty to generate the (a), (b), etc., subfigure
% labels, the optional argument to \subfloat must be present. If a
% subcaption is not desired, just leave its contents blank,
% e.g., \subfloat[].


% An example of a floating table. Note that, for IEEE style tables, the
% \caption command should come BEFORE the table and, given that table
% captions serve much like titles, are usually capitalized except for words
% such as a, an, and, as, at, but, by, for, in, nor, of, on, or, the, to
% and up, which are usually not capitalized unless they are the first or
% last word of the caption. Table text will default to \footnotesize as
% the IEEE normally uses this smaller font for tables.
% The \label must come after \caption as always.
%
%\begin{table}[!t]
%% increase table row spacing, adjust to taste
%\renewcommand{\arraystretch}{1.3}
% if using array.sty, it might be a good idea to tweak the value of
% \extrarowheight as needed to properly center the text within the cells
%\caption{An Example of a Table}
%\label{table_example}
%\centering
%% Some packages, such as MDW tools, offer better commands for making tables
%% than the plain LaTeX2e tabular which is used here.
%\begin{tabular}{|c||c|}
%\hline
%One & Two\\
%\hline
%Three & Four\\
%\hline
%\end{tabular}
%\end{table}


% Note that the IEEE does not put floats in the very first column
% - or typically anywhere on the first page for that matter. Also,
% in-text middle ("here") positioning is typically not used, but it
% is allowed and encouraged for Computer Society conferences (but
% not Computer Society journals). Most IEEE journals/conferences use
% top floats exclusively. 
% Note that, LaTeX2e, unlike IEEE journals/conferences, places
% footnotes above bottom floats. This can be corrected via the
% \fnbelowfloat command of the stfloats package.




\section{Conclusão}
Neste trabalho foi feito o estudo da aplicação de algoritmos genéticos para o problema das N-Rainhas e minimização da função de reastrigin. Foi possível implementar, executar e comparar diversas instâncias de algoritmos genéticos na solução dos dois problemas, de forma a entender como os principais hiper-parâmetros afetam o desempenho do otimizador e os desafios na sua implementação.

Uma grande dificuldade enfrentada durante o trabalho foi como desenhar os experimentos a serem realizados para se avaliar deferentes configurações. É necessário tomar muito cuidado pois pequenos ajustes em alguma cofiguração e inconsistências na definição de hiperparâmetros podem gerar resultados inespreados, necessitando que eles sejam executados novamente. De fato, um ponto de melhora pessoal é estudar mais a literatura de computação evolucionária e entender melhor a metodologia geralmente utilizada pelos autores. Além disso, o uso de controle e ajuste automático de hiper-parâmetros se mostra uma opção muito melhor para ser utilizada nos próximos trabalhos, já que não foram muito exploradas neste primeiro.

% conference papers do not normally have an appendix



% trigger a \newpage just before the given reference
% number - used to balance the columns on the last page
% adjust value as needed - may need to be readjusted if
% the document is modified later
%\IEEEtriggeratref{8}
% The "triggered" command can be changed if desired:
%\IEEEtriggercmd{\enlargethispage{-5in}}

% references section

% can use a bibliography generated by BibTeX as a .bbl file
% BibTeX documentation can be easily obtained at:
% http://mirror.ctan.org/biblio/bibtex/contrib/doc/
% The IEEEtran BibTeX style support page is at:
% http://www.michaelshell.org/tex/ieeetran/bibtex/
%\bibliographystyle{IEEEtran}
% argument is your BibTeX string definitions and bibliography database(s)
%\bibliography{IEEEabrv,../bib/paper}
%
% <OR> manually copy in the resultant .bbl file
% set second argument of \begin to the number of references
% (used to reserve space for the reference number labels box)
\begin{thebibliography}{1}

\bibitem{eibenbook}
Eiben, A.E. and Smith, J.E., 2015. Introduction to evolutionary computing. springer.

\bibitem{bell2009survey}
Bell, J. and Stevens, B., 2009. A survey of known results and research areas for n-queens. Discrete Mathematics, 309(1), pp.1-31.

\bibitem{hoffmeister1991}
Hoffmeister, F. and Bäck, T., 1990, October. Genetic algorithms and evolution strategies: Similarities and differences. In International Conference on Parallel Problem Solving from Nature (pp. 455-469). Springer, Berlin, Heidelberg.


\end{thebibliography}




% that's all folks
\end{document}


